% !TEX root = ../MA119-Question-Bank.tex



% \pgfmathdeclarerandomlist{varx}{{x}{y}{p}{q}{z}}
\pgfmathdeclarerandomlist{varx}{{x}}
\pgfmathrandomitem{\varx}{varx}
\edef\varx{\varx}

\pgfmathtruncatemacro{\aint}{random(3,2)}
\pgfmathtruncatemacro{\d}{random(1,7)}

\pgfmathtruncatemacro{\a}{ifthenelse(\aint!=4, \aint, \aint+1)}


\pgfmathtruncatemacro{\b}{-\a*\d}

\pgfmathtruncatemacro{\B}{-2*\d-\a}
\pgfmathtruncatemacro{\C}{\d^2-\b}


\pgfmathtruncatemacro{\xone}{\d}
\pgfmathtruncatemacro{\xtwo}{\d+\a}

Solve the following radical equation
\[\sqrt{\a \varx \constdisplay{\b} }+\varx=\d.\]

\begin{solution}
\[
  \begin{split}
    \sqrt{\a \varx \constdisplay{\b}}+\varx &=\d\\
    \sqrt{\a \varx \constdisplay{\b}} &=\d-\varx\\
    \a \varx \constdisplay{\b}&=(\varx-\d)^2\\
     \a \varx \constdisplay{\b}&= \varx^2-2\cdot\d\cdot\varx+\d^2\\
    \varx^2 \constdisplay{\B}\varx+\C &=0\\
  \end{split}
\]
\begin{alignat*}{3}
  \varx - \xone&=0 && \ctc{or} & \varx -\xtwo &=0\\
    x&=\xone && \ctc{or} & x &=\xtwo
\end{alignat*}
Check the solutions.

When $\varx=\xone$, the left hand side of the equation equals
 \[\sqrt{\a\cdot\xone\constdisplay{\b} }+\d =\sqrt{0}+\d =\d.\]
So $\varx=\xone$ is a solution.

When $\varx=\xtwo$, the left hand side of the equation equals
 \[\sqrt{\a\cdot\xtwo\constdisplay{\b} }+\d =\sqrt{\a^2}+\d=\a+\d \neq \d.\]
So $\varx=\xtwo$ is not a solution.

Therefore, the equation has only one solution $x=\xone$.
\end{solution}
