% !TEX root = ../MA119-Question-Bank.tex



\pgfmathtruncatemacro{\a}{-random(1,3)} 
\pgfmathtruncatemacro{\v}{-random(4,1)+2}
%%%%% To define a function $f(x)=(x-a)^2-v$

% \pgfmathtruncatemacro{\absv}{abs(\v)}



\pgfmathtruncatemacro{\m}{random(2,1)-1}


\pgfmathtruncatemacro{\x}{\a+\m}
\pgfmathtruncatemacro{\y}{\m^2+\v}


\pgfmathtruncatemacro{\s}{random(2,4)}
\pgfmathtruncatemacro{\t}{\v+random(1,3)}


\pgfmathtruncatemacro{\dotshape}{random(1,2)-1}


Use the graph on the right to answer the following questions. 

\vspace{-\baselineskip}

\begin{minipage}[t]{\textwidth}
\begin{minipage}[c]{0.6\textwidth}
\begin{enumerate}[label={\textup(\arabic*)},afterlabel=~~~]
\item Determine whether the graph is a function and explain your answer.
\item Find the domain of the graph (write the domain in interval notation).
\item Find the range of the graph (write the range in interval notation).
\end{enumerate}
\end{minipage}\quad\quad
\begin{minipage}[c]{0.35\textwidth}
\vspace{4ex}
\begin{tikzpicture}[scale=0.75]
\begin{axis}[
 grid=both, 
 ymin=-5,ymax=5,
 xmax=5,xmin=-5,
 % xtick={-5,-4,...,5},
 % ytick={-5,-4,...,5},
 minor tick num=1,
]
\addplot[thick, restrict x to domain=-5:\x, restrict y to domain=-5:5, name path=A, stealth-]   {(x-\a)^2+\v};           
\addplot[thick, draw] ({\s-0.01},\t)--(\x, \y);
% \node[draw,shape=circle, minimum size=0.25mm,inner sep=0pt,outer sep=0pt,fill=black] at (\x,\y) {};
\ifnum\dotshape=0
\node[draw,shape=circle, minimum size=1.25mm,inner sep=0pt,outer sep=0pt] at (\s,\t) {}; 
\else
\node[draw,shape=circle, minimum size=1.25mm,inner sep=0pt,outer sep=0pt,fill=black] at (\s,\t) {};
\fi  
\node[draw,shape=circle, minimum size=1.25mm,inner sep=0pt,outer sep=0pt,fill=black] at (\a,\v) {};           
\end{axis}
\end{tikzpicture}
\end{minipage}
\end{minipage}


\begin{solution}\mbox{}
\begin{enumerate}[label={\textup(\arabic*)},afterlabel=~~~]
\item The graph is a function. Because every vertical line intersects with the graph at most one point.
\item The domain is $(-\infty, \s \ifnum\dotshape=0)\else ]\fi$.
\item The range is $\ifnum\t<\v \ifnum\dotshape=0 (\t \else [\t \fi \else [\v \fi, +\infty)$.
\end{enumerate}
\end{solution}