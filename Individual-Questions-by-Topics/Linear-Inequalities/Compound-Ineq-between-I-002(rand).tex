% !TEX root = ../MA119-Question-Bank.tex



\pgfmathtruncatemacro{\a}{random(2,8)} %%%% left bound 
\pgfmathtruncatemacro{\b}{random(2,9)}    %%%% right bound
\pgfmathtruncatemacro{\c}{random(5,8)}



\pgfmathtruncatemacro{\d}{random(1,11) }
\pgfmathtruncatemacro{\e}{-random(3,12)}




\pgfmathtruncatemacro{\f}{-random(1,11)}
\pgfmathtruncatemacro{\g}{random(2,5)}



\pgfmathtruncatemacro{\lden}{\c/gcd((\a*\d-\b-\c*\f),\c)}
\pgfmathtruncatemacro{\lnum}{(\a*\d-\b)/gcd((\a*\d-\b-\c*\f),\c)}



\pgfmathtruncatemacro{\rden}{\c/gcd((\a*\e-\b-\c*\g),\c)}
\pgfmathtruncatemacro{\rnum}{(\a*\e-\b)/gcd((\a*\e-\b-\c*\g),\c)}


\pgfmathtruncatemacro{\lclear}{\lnum*\c/\lden}
\pgfmathtruncatemacro{\rclear}{\rnum*\c/\rden}

\pgfmathtruncatemacro{\ll}{\lclear+\b}
\pgfmathtruncatemacro{\rr}{\rclear+\b}


% \pgfmathtruncatemacro{\s}{\m-\d}
% \pgfmathtruncatemacro{\t}{\n+\f}




Solve the compound linear inequality 
\[\ifnum\lden=1 \lnum \else \frac{\lnum}{\lden}\fi > \frac{\a x-\b}{\c} > \ifnum\rden=1 \rnum \else \frac{\rnum}{\rden}\fi.\]



\begin{solution}
\begin{center}
\begin{alignat*}{2}
\ifnum\lden=1 \lnum \else \frac{\lnum}{\lden}\fi
 > &&\centermath{ \frac{\a x-\b}{\c} }& > 
\ifnum\rden=1 \rnum \else \frac{\rnum}{\rden}\fi\\
\lclear > && \centermath{ \a x-\b }& >\rclear \\
\ll> &&\centermath{ \a x } & >\rr\\
\d> &&\centermath{ x } & >\e \\
\end{alignat*}
\end{center}

So the solution set is $(\e, \d)$.

\end{solution}
