% !TEX root = ../MA119-Question-Bank.tex



\pgfmathdeclarerandomlist{varx}{{x}{y}{s}{t}}
% \pgfmathdeclarerandomlist{varx}{{x}{y}}
\pgfmathrandomitem{\varx}{varx}
\edef\varx{\varx}



\pgfmathsetmacro{\mm}{int(random(1,5))}
\pgfmathsetmacro{\initn}{int(random(2,5))}
\pgfmathtruncatemacro{\nn}{\initn/gcd(\mm,\initn)}

\pgfmathsetmacro{\aa}{int(1)}

%%%%%% generate random b coprime with a %%%
\pgfmathsetmacro{\bb}{int(random(1,2))}
\pgfmathtruncatemacro{\bb}{ifthenelse(gcd(\aa, \bb)==1, \bb, \bb/gcd(\aa,\bb) )}
%%%%%%%%%%%%%%%%%%%%%%%%%%%%%%%%%%%%%%%%%%%%

\pgfmathsetmacro{\cc}{int(1)}


 %%%%%% generate random d coprime c %%%
\pgfmathsetmacro{\initd}{int(random(5,3))}
\pgfmathtruncatemacro{\dd}{ifthenelse(gcd(\cc, \initd)==1, \initd, \initd/gcd(\cc,\initd) )}
%%%%%%%%%%%%%%%%%%%%%%%%%%%%%%%%%%%%%%%%%%%%


 \pgfmathtruncatemacro{\AA}{\aa*\cc}
 \pgfmathtruncatemacro{\CC}{(\bb)*(-\dd)}
 \pgfmathtruncatemacro{\bc}{(\bb)*\cc}
 \pgfmathtruncatemacro{\ad}{\aa*(-\dd)}
 \pgfmathtruncatemacro{\BB}{\ad+\bc}

\pgfmathtruncatemacro{\MA}{\aa*\mm}
\pgfmathtruncatemacro{\MB}{-\aa*\nn+\bb}

\pgfmathtruncatemacro{\MC}{\cc*\mm}
\pgfmathtruncatemacro{\MD}{-\cc*\nn-\dd}

\pgfmathtruncatemacro{\MBC}{\MB*\MC}
\pgfmathtruncatemacro{\MAD}{\MA*\MD}


\pgfmathtruncatemacro{\SA}{\MA*\MC}
\pgfmathtruncatemacro{\SB}{\MA*\MD+\MB*\MC}
\pgfmathtruncatemacro{\SC}{\MB*\MD}


Factor completely the following polynomial.
\[\invisibleoneincoef{\AA} (\mm\varx-\nn)^2  \plusminus{\BB}(\mm\varx-\nn) \ifnum\CC>0 +\CC\else \CC\fi.\]

\begin{solution}

\begin{multicols}{2}

\[
\begin{split}
  &\invisibleoneincoef{\AA} (\mm\varx-\nn)^2  \plusminus{\BB}(\mm\varx-\nn) \ifnum\CC>0 +\CC\else \CC\fi\\
    =&(\invisibleoneincoef{\aa}(\mm\varx-\nn)+\bb)(\invisibleoneincoef{\cc}(\mm\varx-\nn)-\dd)\\
    =&\ifnum\MB=0 \invisibleoneincoef{\MA}\varx \else(\invisibleoneincoef{\MA}\varx \constdisplay{\MB})\fi (\invisibleoneincoef{\MC}\varx \constdisplay{\MD}).
\end{split}
\]
\columnbreak

\begin{tikzpicture}
 \matrix (m) [
 matrix of math nodes,
row sep=-\pgflinewidth,
 column sep=-.5\pgflinewidth,
 minimum width=2em,
 ]
 { A=\AA &{}&C=\CC & \\
   \aa &{}&\bb & \\
   \cc &{}&-\dd &  \\
    \bc& + & \ad &=\BB=B\\
 };
 \path
   (m-2-1) edge (m-3-3)
   (m-2-3) edge (m-3-1); 
 \draw[red] (m-4-1.north west) -- (m-4-4.north east);  
\end{tikzpicture}
\end{multicols}

\end{solution}


\begin{solution}
This is another way to factor the polynomial. We simplify the polynomial first and then factor.
% \begin{multicols}{2}

\[
\begin{split}
  &\invisibleoneincoef{\AA} (\mm\varx-\nn)^2  \plusminus{\BB}(\mm\varx-\nn) \ifnum\CC>0 +\CC\else \CC\fi\\
    =&\invisibleoneincoef{\SA}\varx^2  \ifnum\SB=0\else\constdisplay{\SB}\varx\fi \ifnum\SC=0\else\constdisplay{\SC} \fi\\
    =&{\ifnum\MB=0 \invisibleoneincoef{\MA}\varx \else(\invisibleoneincoef{\MA}\varx \constdisplay{\MB})\fi} 
      {\ifnum\MD=0 \invisibleoneincoef{\MC}\varx \else(\invisibleoneincoef{\MC}\varx \constdisplay{\MD})\fi}.
\end{split}
\]
% \columnbreak

% \begin{tikzpicture}
%  \matrix (m) [
%  matrix of math nodes,
% row sep=-\pgflinewidth,
%  column sep=-.5\pgflinewidth,
%  minimum width=2em,
%  ]
%  { A=\SA &{}&C=\SC & \\
%    \MA &{}&\MB & \\
%    \MC &{}&\MD &  \\
%     \display{\MBC} & + & \display{\MAD} &=\SB=B\\
%  };
%  \path
%    (m-2-1) edge (m-3-3)
%    (m-2-3) edge (m-3-1); 
%  \draw[red] (m-4-1.north west) -- (m-4-4.north east);  
% \end{tikzpicture}
% \end{multicols}

\end{solution}