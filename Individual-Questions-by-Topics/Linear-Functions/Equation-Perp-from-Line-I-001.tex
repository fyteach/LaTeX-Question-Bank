% !TEX root = ../MA119-Question-Bank.tex



\pgfmathtruncatemacro{\aa}{random(1,3)} 
\pgfmathtruncatemacro{\bb}{random(3,1)}
\pgfmathtruncatemacro{\ccinit}{random(1,7)-4}
\pgfmathtruncatemacro{\cc}{ifthenelse(\ccinit==0,\ccinit+1,\ccinit)}

\pgfmathtruncatemacro{\mtop}{\aa*random(1,2)}


\pgfmathsetmacro{\mm}{\mtop/\bb}

\pgfmathtruncatemacro{\xinit}{\bb/gcd(\mtop,\bb)}

\pgfmathtruncatemacro{\xx}{ifthenelse(\cc<0,\xinit,-\xinit)}

\pgfmathtruncatemacro{\mmyy}{\mtop/gcd(\mtop,\bb)}

\pgfmathtruncatemacro{\yy}{ifthenelse(\cc<0, \mmyy+\cc, -\mmyy+\cc)}

% \begin{multicols}{2}
	Find an equation of the line passing though $(0, \cc)$ and perpendicular to the line in the graph.
\vspace{-\baselineskip}
	% \columnbreak
\begin{center}
	\begin{flushright}
		\begin{tikzpicture}[every node/.style={scale=1.25}]
			\begin{axis}[
				grid=both,
				minor tick num=1,
				xmax=6,
				xmin=-6,
				ymax=6,
				ymin=-6,
				color=black
			]
			\addplot [
				domain=-6:6, 
				samples=100, 
				color=blue,
				]
				{\mm*x+\cc};
			% \draw[fill=black] (0,0) circle (2pt) node[below left, xshift=1pt, yshift=1pt]{\tiny $0$};
			\draw[fill=black] (0,\cc) circle (2pt) ;
			\draw[fill=black]  (\xx,\yy) circle (2pt) ;
			\end{axis}
			\end{tikzpicture}
	\end{flushright}

\end{center}
% \end{multicols}

\begin{solution}
Since the $y$-intercept is $(0, \cc)$ and the point $(\xx, \yy)$ is also on the line, the slope of the line is  
\[
m=\frac{\yy-\display{\cc}}{\xx-0}=\rdfrac{\mtop}{\bb}.
\]

The slope of the perpendicular line is $m_\perp=-\frac{1}{m}=-\rdfrac{\bb}{\mtop}$.

Then the slope-intercept form equation of the perpendicular line is
\[
	y=-\rdfrac{\bb}{\mtop} x \constdisplay{\cc}.
\]
\end{solution}